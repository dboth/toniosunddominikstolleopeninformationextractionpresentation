\documentclass[11pt]{beamer}
\usetheme{OIE}
\usepackage[utf8]{inputenc}
\usepackage[german]{babel}
\usepackage[T1]{fontenc}
\usepackage{amsmath}
\usepackage{amsfonts}
\usepackage{amssymb}
\usepackage{tikz}
\usepackage{graphicx}
\usepackage{color}
\usepackage{booktabs}
\usepackage{pifont}
\usepackage{xcolor}
\usepackage[natbib=true,backend=bibtex,style=authoryear]{biblatex}
\author{Dominik Both, Tonio Weidler}
\title{Open Information Extraction}
%\setbeamercovered{transparent} 
%\setbeamertemplate{navigation symbols}{} 
%\logo{} 
\institute{Institut für Computerlinguistik, Universität Heidelberg} 
\date{15.07.2016} 
\subject{} 
\definecolor{lightgray}{gray}{0.8}

\begin{document}

% AUTOMATISMEN
\AtBeginSection{\frame{\sectionpage}}
\AtBeginSubsection{\frame{\subsectionpage}}


% TEMPLATING
\setbeamertemplate{title page}{
	\center 
		\begin{beamercolorbox}[center]{part title}
	      \Huge\inserttitle\par
	    \end{beamercolorbox}
	    \vspace{15pt}
		\insertauthor\\
		\vspace{15pt}		
		Proseminar \textit{Text Mining}\\
		Andrea Zielinski\\
		\vspace{15pt}
		\insertinstitute, \insertdate
}

\defbeamertemplate{section page}{wiqe}[1][]{%
  \begin{centering}
    \begin{beamercolorbox}[center]{part title}
      \Huge\insertsection\par
    \end{beamercolorbox}
  \end{centering}
}

\defbeamertemplate{subsection page}{wiqe}[1][]{%
  \begin{centering}
    \begin{beamercolorbox}[center]{}
      \usebeamerfont{subsection title}\usebeamercolor[fg]{subsection name}\insertsection\par
    \end{beamercolorbox}
    \begin{beamercolorbox}[sep=2pt,center]{part title}
		\huge\insertsubsection\par
    \end{beamercolorbox}
  \end{centering}
}
\setcounter{tocdepth}{1}
\setbeamertemplate{section page}[wiqe]
\setbeamertemplate{subsection page}[wiqe]

% COMMANDS
\newcommand{\hitem}{
	\item[\color{lightgray}\rule{0.5em}{0.5em}]
}

\begin{frame}
\titlepage
\end{frame}

\begin{frame}{Strukturierung}
    \tableofcontents
\end{frame}

\section{Einleitung}
\section{OIE - Grundlagen}
	\subsection{Motivation und Hürden}
	\subsection{Verfahren}
	\subsection{Datenrepräsentationsformen}
		\begin{frame}{Standardpattern}
			\begin{center}
				\includegraphics[scale=0.5]{img/oie-pattern.png}\\
				\vspace{15pt}
				\textbf{Argument A} geht eine gerichtete \textbf{Relation} mit \textbf{Argument B} ein.
			\end{center}
		\end{frame}
		
		\begin{frame}{Unnormierte Annotation}
			\begin{center}
				(argument\_a, predicate\_x, argument\_b)\\
				(argument\_a, predicate\_y, argument\_c)\\
				(argument\_a, predicate\_y, argument\_d)
			\end{center}
			\vspace{15pt}
			\textbf{Probleme}\\
			\begin{itemize}
				\item redundant
				\item unnormiert
				\item kann nur zweistellige Relationen ausdrücken
			\end{itemize}
		\end{frame}
		
		\begin{frame}{RDF und Linked Data}
			\begin{block}{Resource Description Framework}
				Modelliert Aussagen (triples) über \textbf{Subjekte}, \textbf{Objekte} und \textbf{Prädikate}, in denen das Objekt das Subjekt anhand einer Relation (dem Prädikat) genauer beschreibt. Dabei entseht ein gerichteter Graph.
			\end{block}
			\vspace{10pt}
			\begin{center}
				:subject :predicate :object.
				\vspace{10pt}
				\includegraphics[scale=0.5]{img/oie-rdf-triple.png}
			\end{center}
		\end{frame}
		
		\begin{frame}{RDF Konzepte und Notationen}
			\begin{itemize}
				\item \textbf{URIs}\\
					macht Ressourcen (S, R, O) eindeutig identifizierbar und verweist auf zusätzlich Informationen
				\item \textbf{Interpretierbar}
					erlaubt Schlussfolgerungen nach definierten Regeln
				\item \textbf{Turtle}
					Syntax erlaubt verkürzte Schreibweisen
			\end{itemize}
		\end{frame}
		
		\begin{frame}[fragile]{RDF Syntax}
			\begin{verbatim}
				dbr:Barack_Obama a foaf:person, :President;
				    dbo:spouse dbr:Michelle_Obama.
				dbr:Bernie_Sanders dbo:birthPlace dbr:New_York, 
				                                  dbr:Brooklyn;
				dbr:Brooklyn dbo:isPartOf dbr:New_York 	
			\end{verbatim}
		\end{frame}
		
		\begin{frame}{... als Graph}
			\begin{center}
				\includegraphics[scale=0.4]{img/oie-rdf-example.png}
			\end{center}			
		\end{frame}
\section{Anwendungsbeispiel: LODifier}
	\subsection{Struktur}
	\subsection{Preprocessing}
	\subsection{RDF Construction}
	\subsection{Bewertung}
\section{LODifier und andere Systeme im Kontext}
	\subsection{Vergleich}
	\subsection{Bewertung der Ansätze}
\section{Fazit und Ausblick}
	\subsection{Probleme und Hürden}
	\subsection{Entwicklungsmöglichkeiten}

\end{document}